	\setlength\LTleft{-0.75in}
%	\setlength\LTright{-1in}
%\small
\footnotesize
    \begin{longtable}{C{2.1cm} C{2.1cm} C{2.3cm} C{1.5cm} C{2.7cm} C{2.2cm} }
    %\begin{longtable}{|C{2cm}|c|C{2.5cm}|C{3cm}|C{2.4cm}|c|}

		\hline
    	\textbf{Authors} & \textbf{Preprocessing} & \textbf{Segmentation} & \textbf{Features} & \textbf{Classification} & \textbf{Performance}  \\[1pt] \hline

    Ahirwar \emph{et al.}, 2012  &
    	- &
    	thresholding + granulometry, opening, morphological gradient, dilation, closing, thinning, spur removal &
    	- &
    	five (\emph{P.falciparum}, \emph{P.vivax}, \emph{P.ovale}, \emph{P.malariae} infected, and noninfected) &
    	- \vspace{0.6cm} \\
    	%\hline

    Anggraini \emph{et al.}, 2011 &
		- &
		thresholding + hole filling &	
		- &
		two (\emph{P.falciparum} infected and noninfected) + two life-cycle-stages &
		SE=93\% SP=99\%
        \vspace{0.6cm}
		\\
    Arco \emph{et al.}, 2014 &
		- &
		adaptive thresholding + hole filling, closing, regional minima &	
		- &
		two (infected and noninfected) &
		Acc=96.46\%
		\vspace{0.6cm}
        \\
    Das \emph{et al.}, 2011 &
		- &
		marker controlled watershed &	
		opening, closing  &
		two (infected and noninfected) &
		Acc=88.77\%
		\vspace{0.6cm}
        \\

    Das \emph{et al.}, 2013 &
		- &
		marker controlled watershed &	
		- &
	three (\emph{P.falciparum}, \emph{P.vivax} infected and noninfected) + three life-cycle-stages per species &
		Acc=84\%
		\vspace{0.6cm}
        \\

    Das \emph{et al.}, 2014 &
		- &
		marker controlled watershed &	
		- &
	three (\emph{P.falciparum}, \emph{P.vivax} infected and noninfected) + three life-cycle-stages per species &
		SE=99.72\% SP=84.39\%
		\vspace{0.6cm}
        \\

    Dave \emph{et al.}, 2017 &
		- &
		adaptive thresholding + erosion, dilation &	
		- &
		two (infected and noninfected) &
		Acc=97.83\% thin films, Acc=89.88\% thick films
		\vspace{0.6cm}
        \\

    Devi \emph{et al.}, 2017 &
		- &
		marker controlled watershed &	
		- &
		two (infected and noninfected)  &
		Acc=98.02\%
		\vspace{0.6cm}
        \\

    Diaz \emph{et al.}, 2009 &
		- &
		inclusion tree &	
		- &
		two (\emph{P.falciparum} infected and noninfected) + three life-cycle-stages &
		SE=94\% SP=99.7\% for detection, SE=78.8\% SP=91.2\% for life-stages
\vspace{0.6cm}
		\\

    Di Ruberto \emph{et al.}, 2002 &
		area closing &
		thresholding + granulometry, watershed  transform &	
		skeleton &
		two (\emph{P.falciparum} infected and noninfected) + three life-cycle-stages &
		-
\vspace{0.6cm}
		\\

    Elter \emph{et al.}, 2011 &
		- &
		thresholding + black top-hat, dilation &	
		- &
		two (infected and noninfected) &
		SE=97\%
		\vspace{0.6cm} \\

    Gonzalez-Betancourt \emph{et al.}, 2016  &
    	morphological filter, erosion-reconstruction, dilation-reconstruction, closing &
    	-  &
    	- &
    	- &
    	- \vspace{0.6cm} \\	

    Ghosh \emph{et al.}, 2011  &
    	- &
    	thresholding + opening, closing  &
    	- &
    	two (\emph{P.vivax} infected and noninfected) &
    	- \vspace{0.6cm} \\

    Kareem \emph{et al.}, 2012  &
    	dilation, erosion &
    	-  &
    	- &
    	two (infected and noninfected) &
    	Acc=88\% SE=90\% SP=86\% \vspace{0.6cm} \\	

    Khan \emph{et al.}, 2011  &
    	area closing &
    	thresholding + granulometry, opening, morphological reconstruction, gradient, dilation &
    	- &
    	five (\emph{P.falciparum}, \emph{P.vivax}, \emph{P.ovale}, \emph{P.malariae} infected, and noninfected) &
    	Acc=81\% SE=85.5\% \vspace{0.6cm} \\	

    Khan \emph{et al.}, 2014  &
    	- &
    	- &
    	- &
    	two (infected and noninfected) &
    	SE=93\% SP=95\% \vspace{0.6cm} \\	

    Malihi \emph{et al.}, 2013  &
    	closing &
    	-  &
    	area granulometry &
    	two (infected and noninfected) &
    	Acc=91\% SE=80\% SP=95.5\% \vspace{0.6cm} \\	

    Mushabe \emph{et al.}, 2013  &
    	closing &
    	thresholding + granulometry, dilation, erosion  &
    	- &
    	two (infected and noninfected) &
    	SE=98.5 SP=97.2\% \vspace{0.6cm} \\	

    Oliveira \emph{et al.}, 2017  &
    	erosion &
    	-  &
    	- &
    	two (infected and noninfected) &
    	Acc=91\% \vspace{0.6cm} \\

    Reni \emph{et al.}, 2015  &
    	new morphological filtering &
    	-  &
    	- &
    	- &
    	- \vspace{0.6cm} \\

    Romero-Rondon \emph{et al.}, 2016  &
    	opening &
    	marker controlled watershed, erosion  &
    	- &
    	- &
    	- \vspace{0.6cm} \\

    Rosado \emph{et al.}, 2017  &
    	- &
    	adaptive thresholding + closing  &
    	- &
    	four (\emph{P.falciparum}, \emph{P.ovale}, \emph{P.malariae} infected, and noninfected) +
    three life-cycle-stages for species &
    	SE=73.9-96.2\% SP=92.6-99.3\% \vspace{0.6cm} \\

    Ross \emph{et al.}, 2006  &
    	area closing &
    	thresholding + granulometry, opening, reconstruction, morphological gradient, closing, thinning  &
        - &
    	five (\emph{P.falciparum}, \emph{P.vivax}, \emph{P.ovale}, \emph{P.malariae} infected, and noninfected) &
    	SE=85\% for detection, Acc=73\% for classification
    	 \vspace{0.6cm} \\


    Savkare \emph{et al.}, 2011a  &
    	- &
    	thresholding + hole filling, watershed transform  &
    	- &
    	two (infected and noninfected) &
    	- \vspace{0.6cm} \\

    Savkare \emph{et al.}, 2011b  &
    	- &
    	thresholding + hole filling, watershed transform  &
    	- &
    	two (infected and noninfected) &
    	SE=93.12\% SP=93.17\% \vspace{0.6cm} \\

    Savkare \emph{et al.}, 2015  &
    	- &
    	thresholding + watershed transform, erosion, dilation  &
    	- &
    	two (infected and noninfected) &
    	Acc=95.5\% \vspace{0.6cm} \\

    Sheikhhosseini \emph{et al.}, 2013  &
    	- &
    	thresholding + hole filling, opening  &
    	- &
    	two (infected and noninfected) &
    	Acc=97.25\% SE=82.21\% SP=98.02\% \vspace{0.6cm} \\

    Somasekar \emph{et al.}, 2015  &
    	- &
    	fuzzy C-means clustering + erosion, hole filling  &
    	- &
    	two (infected and noninfected) &
    	SE=98\% SP=93.3\% \vspace{0.6cm} \\

    Somasekar \emph{et al.}, 2017  &
    	- &
    	thresholding + erosion, closing, hole filling  &
    	- &
    	two (infected and noninfected) &
    	average DSC=0.8 \vspace{0.6cm} \\

    Soni \emph{et al.}, 2011  &
    	- &
    	thresholding + granulometry, morphological gradient, dilation  &
    	- &
    	five (\emph{P.falciparum}, \emph{P.vivax}, \emph{P.ovale}, \emph{P.malariae} infected, and noninfected) &
    	SE=98\% for detection  \vspace{0.6cm} \\

{\v{S}}pringl, 2009  &
    	closing &
    	thresholding + marker controlled watershed transform, hole filling, dilation, opening, erosion  &
    	- &
    	two (infected and noninfected) &
    	AUC=0.98 \vspace{0.6cm} \\

    Sulistyawati \emph{et al.}, 2015  &
    	- &
    	blob analysis + erosion, dilation, opening, closing, hole filling  &
    	- &
    	two (infected and noninfected) &
    	Acc=99.39\% \vspace{0.6cm} \\

 Tek \emph{et al.}, 2006  &
    	- &
    	top-hat, infinite reconstruction  &
    	area granulometry &
    	two (infected and noninfected) &
    	SE=74\% SP=98\% \vspace{0.6cm} \\

    Tek \emph{et al.}, 2010  &
    	closing, granulometry &
    	thresholding + granulometry, area top-hat, closing  &
    	area granulometry &
    	five (\emph{P.falciparum}, \emph{P.vivax}, \emph{P.ovale}, \emph{P.malariae} infected, and noninfected) +
    four life-cycle-stages for species &
    	SE=72\% SP=98\% \vspace{0.6cm} \\ 

    Yunda \emph{et al.}, 2012  &
    	- &
    	thresholding + morphological gradient, erosion, dilation  &
    	- &
    	three (\emph{P.falciparum}, \emph{P.vivax} infected, and noninfected) + two life-cycle-stages for \emph{P.falciparum}&
    	SE=77.19\% \\
    	  	
		 \hline
		
				
	\caption{Summary of analysed methods: morphological operations used in the main phases of analysis, kind of classification and performance measures (Sensitivity, Specificity, Accuracy, if reported).} % needs to go inside longtable environment
	\label{tab:summary_table}
	\end{longtable}

	